\begin{table}[H]
\centering
\captionof{table}[Anhang]{Titel}
%\caption{Zuordnung der Items}
\label{itemtabelle}
\resizebox{\textwidth}{!}{%
\begin{tabular}{@{}lllll@{}}
\toprule
Item  & Konstrukt                                                                    & Indikator                                                                                                                                                                           & Ausprägung                                                                                                                                                                                       & Skalenniveau \\ \midrule
1-5   & \begin{tabular}[c]{@{}l@{}}Fach-\\ kompetenz\\FK1,FK2,FK3,FK4,FK5\end{tabular}                    & \begin{tabular}[c]{@{}l@{}}Disposition sachlich-gegenständliche\\ Probleme selbstorganisiert lösen zu können\end{tabular}                                                           & \begin{tabular}[c]{@{}l@{}}5-stufige Likertskala:\\ 1= trifft überhaupt nicht zu\\ 2= trifft wenig zu\\ 3=trifft teils/teils zu\\ 4=trifft überwiegend zu\\ 5=trifft völlig zu\end{tabular}      & metrisch     \\
\midrule
6-10  & \begin{tabular}[c]{@{}l@{}}Methoden-\\ kompetenz\\MK1,MK2,MK3,MK4\end{tabular}                & \begin{tabular}[c]{@{}l@{}}Tätigkeiten und Aufgaben methodisch\\  selbst-organisiert zu gestalten und Methoden\\  weiter zu entwickeln\end{tabular}                                 & \begin{tabular}[c]{@{}l@{}}5-stufige Likertskala: \\ 1= trifft überhaupt nicht zu\\ 2= trifft wenig zu \\ 3=trifft teils/teils zu \\ 4=trifft überwiegend zu \\ 5=trifft völlig zu\end{tabular}  & metrisch     \\
\midrule
11-16 & \begin{tabular}[c]{@{}l@{}}Personal-\\ kompetenz\\PK1,PK2,PK3,PK4,PK5\end{tabular}                & \begin{tabular}[c]{@{}l@{}}Sich einschätzen, selbstorganisiert reflexiv\\  handeln,Werte, Motive und Selbstbilder\\  entwickeln\end{tabular}                                        & \begin{tabular}[c]{@{}l@{}}5-stufige Likertskala: \\ 1= trifft überhaupt nicht zu \\ 2= trifft wenig zu \\ 3=trifft teils/teils zu \\ 4=trifft überwiegend zu \\ 5=trifft völlig zu\end{tabular} & metrisch     \\
\midrule
17-21 & \begin{tabular}[c]{@{}l@{}}Komunikations-\\ kompetenz\\KK1,KK2,KK3,KK4\end{tabular}           & \begin{tabular}[c]{@{}l@{}}Sich mit anderen kreativ auseinander setzen, \\ kommunikativ und selbstorganisiert handeln\\ \parencite[8]{ErpenbeckRosenstiel200305}\end{tabular} & \begin{tabular}[c]{@{}l@{}}5-stufige Likertskala\\ 1= trifft nicht zu,\\2= trifft wenig zu \\ 3=trifft teils/teils zu \\ 4=trifft überwiegend zu \\ 5=trifft vollständig zu\end{tabular}                                                                                    & metrisch     \\
 \midrule
22-28 & \begin{tabular}[c]{@{}l@{}}Nutzungs\\ verhalten\\ Lernplattform\end{tabular} & \begin{tabular}[c]{@{}l@{}}Aufschluss über die Intensität und Richtung der\\ Nutzung der Lernplattform\end{tabular}                                                                 & \begin{tabular}[c]{@{}l@{}}5-stufige Skala\\ 1=0-1mal genutzt\\ 2=2-4 mal genutzt\\ 3=5-7 mal genutzt\\ 4=8-10 mal genutzt\\ 5=\textgreater10 mal genutzt\end{tabular}                            & metrisch     \\ \bottomrule
\end{tabular}%
}
\end{table}
