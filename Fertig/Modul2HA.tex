\documentclass[12pt, bibliography=totoc]{scrartcl}
\usepackage[headsepline,automark]{scrlayer-scrpage} %Trennlinie an Kopfzeile
%\usepackage{scrheadings}
\clearpairofpagestyles
\lohead{\rightmark}
%\renewcommand{\partmark}[1]{\relax}% \part daran hindern, den Kolumnentitel zu löschen
\ohead[]{\pagemark}
%\ofoot*{\pagemark}
%%Kopfzeile
\usepackage{txfonts} %für times new roman
%\usepackage{helvet} %für arial, dann aber 11pt
\usepackage[a4paper, left=2cm, right=2.5cm]{geometry}
\usepackage[onehalfspacing]{setspace}
%\usepackage{apacite}
\usepackage{wasysym}
\usepackage{rotating}
\usepackage{amsmath}
\usepackage{amssymb}
\usepackage{float}
%\usepackage{caption}
\usepackage[T1]{fontenc}
\usepackage[utf8]{inputenc}
%\usepackage{todonotes}
\usepackage{enumitem}
%\uespackage{caption}
%\usepackage[bf]{caption}
%\renewcommand{\captionfont}{\small\slshape}
%\renewcommand{\figurename}{Abb.}
%\renewcommand{\thefigure}{\arabic{section}.\arabic{figure}}
%\makeatletter \@addtoreset{figure}{section} \makeatother
%\captionsetup[figure]{skip=1pt}
\usepackage{tabularx}
\usepackage{pdfpages}
\usepackage{array}
\usepackage{hyperref}
\usepackage{threeparttable} %fußnoten unterhalb tabelle
\usepackage{booktabs} % fuer schone Tabellen
\usepackage{rotating} % um tabellen auf quer drehen zu koennen http://www.golatex.de/kann-man-tabellen-im-querformat-darstellen-t2003.html
%\newcolumntype{C}[1]{>{\centering\arraybackslash}p{#1}} %Spalten mit fester breite zentriert
%\newcolumntype{L}[1]{>{\raggedright\arraybackslash}p{#1}} %Spalten mit fester breite linksbündig
%\newcolumntype{Y}{>{\small\raggedright\arraybackslash}X}
%\newcolumntype{C}{>{\small\centering\arraybackslash}X}
\usepackage{graphicx}
%\usepackage[german]{babel}
\usepackage{typearea}
%für Randbemerkungen, sehr nützlich:
\usepackage{xargs}                      % Use more than one optional parameter in a new commands
\usepackage[pdftex,dvipsnames]{xcolor}
\usepackage[colorinlistoftodos,prependcaption,textsize=tiny]{todonotes}
\newcommandx{\unsure}[2][1=]{\todo[linecolor=red,backgroundcolor=red!25,bordercolor=red,#1]{#2}}
\newcommandx{\change}[2][1=]{\todo[linecolor=blue,backgroundcolor=blue!25,bordercolor=blue,#1]{#2}}
\newcommandx{\info}[2][1=]{\todo[linecolor=OliveGreen,backgroundcolor=OliveGreen!25,bordercolor=OliveGreen,#1]{#2}}
\newcommandx{\improvement}[2][1=]{\todo[linecolor=Plum,backgroundcolor=Plum!25,bordercolor=Plum,#1]{#2}}
\newcommandx{\thiswillnotshow}[2][1=]{\todo[disable,#1]{#2}}
% erklaerung siehe hier http://tex.stackexchange.com/questions/9796/how-to-add-todo-notes

%hat prima funktioniert:
\usepackage[style=apa,backend=biber]{biblatex}
\usepackage[american,ngerman]{babel}
\DeclareLanguageMapping{ngerman}{ngerman-apa}
\usepackage[babel,german=guillemets]{csquotes}
%\bibliographystyle{apacite}
%nach part fängt section wieder mit eins an alte Gestaltung
%\makeatletter
%\@addtoreset{section}{part}
%\makeatother
%\renewcommand*{\partformat}{\thepart}{}
%\renewcommand*{\partheadmidvskip}{\nobreak\enskip}
%\bibliography{/Users/iNge/Dropbox/Biblio/ingesbibneu}
\bibliography{/Users/iNge/Dropbox/Biblio/library}
%\bibliography{library}




\begin{document}
\renewcommand\finalandcomma{\addcomma}

\begin{titlepage}
\thispagestyle{empty}
\begin{center}
\color{blue}\Large{Fernuniversität Hagen}\\
\end{center}


\begin{center}
\Large{Bildung und Medien: eEducation}
\end{center}
\begin{verbatim}



\end{verbatim}
\begin{center}
\textbf{\Large{Kommentierte Bibliographie zum Thema Online-Lernen}}
\end{center}
\begin{verbatim}

\end{verbatim}
\begin{center}
\textbf{Fakultät Kulturwissenschaften}
\end{center}
\begin{verbatim}










\end{verbatim}

\begin{flushleft}
\begin{tabular}{lll}
\textbf{Studiengang:} & & MA Bildung und Medien: eEducation\\
& & Modul 2: Anwendungsbezogene Bildungsforschung\\
& & \\
& & \\
\textbf{eingereicht von:} & & {\color{magenta} Inge Koch-Meinass \flq{}ingekoch@mac.com\frq{}}\\
& & {\color{magenta}Matrikelnr.: 123456 }\\
& & \\
\textbf{eingereicht am:} & & 06. November 2015\\
& & \\
& & \\
%\textbf{Betreuer:} & & Herr Prof. Dr. J. A. Müller
\end{tabular}
\end{flushleft}

% das ist wohl jetzt das Ende des Dokumentes
\end{titlepage}


%% das Papierformat zuerst
%\documentclass[a4paper, 11pt]{article}

% deutsche Silbentrennung
%\usepackage[ngerman]{babel}
%\usepackage{color} \color{blue}
% wegen deutschen Umlauten
%\usepackage[utf8]{inputenc}

% hier beginnt das Dokument
%\begin{document}

\begin{titlepage}
\thispagestyle{empty}
\begin{center}
\color{blue}\Large{Fernuniversität Hagen}\\
\end{center}


\begin{center}
%\Large{Bildung und Medien: eEducation}
\end{center}
\begin{verbatim}



\end{verbatim}
\begin{center}
\textbf{\Large{Kommentierte Bibliographie zum Thema Online-Lernen}}
\end{center}
\begin{verbatim}

\end{verbatim}
\begin{center}
%\textbf{im Studiengang Wirtschaftsinformatik}
\end{center}
\begin{verbatim}











\end{verbatim}

\begin{flushleft}
\begin{tabular}{lll}
\textbf{Studiengang:} & & MA Bildung und Medien: eEducation\\
& & Modul 2: Anwendungsbezogene Bildungsforschung\\
& & \\
& & \\
\textbf{eingereicht von:} & & {\color{magenta} Inge Koch-Meinass \flq{}ingekoch@mac.com\frq{}}\\
& & {\color{magenta}Matrikelnr.: 9650962 }\\
& & \\
\textbf{eingereicht am:} & & 06. November 2015\\
& & \\
& & \\
%\textbf{Betreuer:} & & Herr Prof. Dr. J. A. Müller
\end{tabular}
\end{flushleft}

% das ist wohl jetzt das Ende des Dokumentes
\end{titlepage}

\listoftodos{Gesammelte Unklarheiten}
\tableofcontents
%\listoftables
\setcounter{page}{1}
%\thispagestyle{empty}
\pagebreak

\section{Evaluation}\label{evaluation}

Wie wird evaluation definiert? Bei \textcite{askun2007web} wird es wie
folgt beschrieben: Jetzt teste ich mal \textcite{blau2012personality}
editorial Wie bin ich da hingeommen?

\section{Einführung und Kontext}\label{einfuxfchrung-und-kontext}

\section{Theorierahmen}\label{theorierahmen}

\subsection{Begriffliche Definitionen oder aktueller
Forschungsstand}\label{begriffliche-definitionen-oder-aktueller-forschungsstand}

Blended Learning, Erwachsenenbildung

\subsection{Beschreibung der zu untersuchenden
Kompetenzen}\label{beschreibung-der-zu-untersuchenden-kompetenzen}

Hier dann alles zum Thema Operationalisierung

\subsection{Inhalte der Weiterbildung}\label{inhalte-der-weiterbildung}

\subsection{Aufbau der Lernplattform}\label{aufbau-der-lernplattform}

\section{Methoden}\label{methoden}

\subsection{Forschungsfrage und
Hypothesen}\label{forschungsfrage-und-hypothesen}

\subsection{Studiendesign}\label{studiendesign}

\begin{itemize}
\tightlist
\item
  Ziel der Evaluation
\item
  formativ
\item
  natürliche Gruppe
\item
  deskriptive Auswertung univariat
\item
  Darstellung der Ergebnisse anhand von Häufigkeiten
\item
  MW, Median, Modal, Streuung
\item
  schriftliche Befragung, vollstandardisiert, alle Teilnehmer,
  Selbseinschätzung
\end{itemize}

\subsection{Messinstrument}\label{messinstrument}

\subsection{Messinstrument}\label{messinstrument-1}

\subsection{Pretest des Fragebogens}\label{pretest-des-fragebogens}

Nach Entwurf des Fragebogens wurde dieser einem Pretest unterzogen. Es
gab 8 Teilnehmer, die durchschnittlcihe Verweildauer betrug ca. 5
Minuten. Die Reliabilität wurde mittels Cronbachs alpha ermittelt
(\textcite{Wassa}) Er ergab für die einzelnen Kompetenzen Werte zwischen
0.72 und 0.98.

\begin{table}[]
\centering
\caption{My caption}
\label{my-label}
\begin{tabular}{@{}lll@{}}
\toprule
Kompetenz               & Cronbachs alpha & Items               \\ \midrule
Fachkompetenz           & 0.84            & FK1,FK2,FK3,FK4,FK5 \\
Methodenkompetenz       & 0.73            & MK1,MK2,MK3,MK4     \\
Personalkompetenz       & 0.91            & PK1,PK2,PK3,PK4,PK5 \\
Kommunikationskompetenz & 0.93            & KK1,KK2,KK3,K4      \\ \bottomrule
\end{tabular}
\end{table}

Value of Cronbach's Alpha for the Questionnaire (Reliability Test)
calculated using online calculator \textcite{Wassa} . The questionnaire
was tested for its reliability and validity using Cronbach Alfa. The
test results are shown in the following table 4.8.1.

\% Please add the following required packages to your document preamble:
\% \usepackage{booktabs}

\% Please add the following required packages to your document preamble:
\% \usepackage{booktabs} \% \usepackage{graphicx}

\begin{table}[]
\centering
\caption{Zuordnung der Items}
\label{itemtabelle}
\resizebox{\textwidth}{!}{%
\begin{tabular}{@{}lllll@{}}
\toprule
Item  & Konstrukt                                                                    & Indikator                                                                                                                                                                           & Ausprägung                                                                                                                                                                                       & Skalenniveau \\ \midrule
1-5   & \begin{tabular}[c]{@{}l@{}}Fach-\\ kompetenz\end{tabular}                    & \begin{tabular}[c]{@{}l@{}}Disposition sachlich-gegenständliche\\ Probleme selbstorganisiert lösen zu können\end{tabular}                                                           & \begin{tabular}[c]{@{}l@{}}5-stufige Likertskala:\\ 1= trifft überhaupt nicht zu\\ 2= trifft wenig zu\\ 3=trifft teils/teils zu\\ 4=trifft überwiegend zu\\ 5=trifft völlig zu\end{tabular}      & metrisch     \\
\midrule
6-10  & \begin{tabular}[c]{@{}l@{}}Methoden-\\ kompetenz\end{tabular}                & \begin{tabular}[c]{@{}l@{}}Tätigkeiten und Aufgaben methodisch\\  selbst-organisiert zu gestalten und Methoden\\  weiter zu entwickeln\end{tabular}                                 & \begin{tabular}[c]{@{}l@{}}5-stufige Likertskala: \\ 1= trifft überhaupt nicht zu\\ 2= trifft wenig zu \\ 3=trifft teils/teils zu \\ 4=trifft überwiegend zu \\ 5=trifft völlig zu\end{tabular}  & metrisch     \\
\midrule
11-16 & \begin{tabular}[c]{@{}l@{}}Personal-\\ kompetenz\end{tabular}                & \begin{tabular}[c]{@{}l@{}}Sich einschätzen, selbstorganisiert reflexiv\\  handeln,Werte, Motive und Selbstbilder\\  entwickeln\end{tabular}                                        & \begin{tabular}[c]{@{}l@{}}5-stufige Likertskala: \\ 1= trifft überhaupt nicht zu \\ 2= trifft wenig zu \\ 3=trifft teils/teils zu \\ 4=trifft überwiegend zu \\ 5=trifft völlig zu\end{tabular} & metrisch     \\
\midrule
17-21 & \begin{tabular}[c]{@{}l@{}}Komunikations-\\ kompetenz\end{tabular}           & \begin{tabular}[c]{@{}l@{}}Sich mit anderen kreativ auseinander setzen, \\ kommunikativ und selbstorganisiert handeln\\ \parencite[8]{ErpenbeckRosenstiel200305}\end{tabular} & \begin{tabular}[c]{@{}l@{}}5-stufige Likertskala\\ 1= trifft nicht zu,\\ 5=trifft vollständig zu\end{tabular}                                                                                    & metrisch     \\
 \midrule
22-28 & \begin{tabular}[c]{@{}l@{}}Nutzungs\\ verhalten\\ Lernplattform\end{tabular} & \begin{tabular}[c]{@{}l@{}}Aufschluss über die Intensität und Richtung der\\ Nutzung der Lernplattform\end{tabular}                                                                 & \begin{tabular}[c]{@{}l@{}}5-stufige Skala\\ 1=0-1mal genutzt\\ 2=2-4 mal genutzt\\ 3=5-7 mal genutzt\\ 4=8-10 mal genutzt\\ 5=\textgreater10 mal genutz\end{tabular}                            & metrisch     \\ \bottomrule
\end{tabular}%
}
\end{table}

\section{Beschreibung des
Lernsettings}\label{beschreibung-des-lernsettings}

Es wird im folgenden das eLearning Angebot einer betrieblichen
Weiterbildung evaluiert \autocite{Helmke2008a}. Diese Umgebung findet
auch bei \textcite{ChenCouse200917} Anwendung. Immer noch zitieren
\autocite[vgl.][pp.~11-15]{Sussman2010} \autocite{Wright2013}
\textcite[30-38]{Goldstandard} \autocite[72]{Kromrey2009}

\pagebreak
\printbibliography
\pagebreak
\appendix


  \section{Anhang}
  %\subsection{Abkürzungen}
  %\renewcommand{\thetable}{{A}.\arabic{table}}
  %\renewcommand{\thetable}{\Alph{section}.\arabic{table}}
  \begin{table}[H]
\centering
%\captionof{table}[Anhang]{Titel}
%\caption{Zuordnung der Items}
\label{itemtabelle}
\resizebox{\textwidth}{!}{%
\begin{tabular}{@{}lllll@{}}
\toprule
Item  & Konstrukt                                                                    & Indikator                                                                                                                                                                           & Ausprägung                                                                                                                                                                                       & Skalenniveau \\ \midrule
1-5   & \begin{tabular}[c]{@{}l@{}}Fach-\\ kompetenz\\FK1,FK2,FK3,FK4,FK5\end{tabular}                    & \begin{tabular}[c]{@{}l@{}}Disposition sachlich-gegenständliche\\ Probleme selbstorganisiert lösen zu können\end{tabular}                                                           & \begin{tabular}[c]{@{}l@{}}5-stufige Likertskala:\\ 1= trifft überhaupt nicht zu\\ 2= trifft wenig zu\\ 3=trifft teils/teils zu\\ 4=trifft überwiegend zu\\ 5=trifft völlig zu\end{tabular}      & metrisch     \\
\midrule
6-10  & \begin{tabular}[c]{@{}l@{}}Methoden-\\ kompetenz\\MK1,MK2,MK3,MK4\end{tabular}                & \begin{tabular}[c]{@{}l@{}}Tätigkeiten und Aufgaben methodisch\\  selbst-organisiert zu gestalten und Methoden\\  weiter zu entwickeln\end{tabular}                                 & \begin{tabular}[c]{@{}l@{}}5-stufige Likertskala: \\ 1= trifft überhaupt nicht zu\\ 2= trifft wenig zu \\ 3=trifft teils/teils zu \\ 4=trifft überwiegend zu \\ 5=trifft völlig zu\end{tabular}  & metrisch     \\
\midrule
11-16 & \begin{tabular}[c]{@{}l@{}}Personal-\\ kompetenz\\PK1,PK2,PK3,PK4,PK5\end{tabular}                & \begin{tabular}[c]{@{}l@{}}Sich einschätzen, selbstorganisiert reflexiv\\  handeln,Werte, Motive und Selbstbilder\\  entwickeln\end{tabular}                                        & \begin{tabular}[c]{@{}l@{}}5-stufige Likertskala: \\ 1= trifft überhaupt nicht zu \\ 2= trifft wenig zu \\ 3=trifft teils/teils zu \\ 4=trifft überwiegend zu \\ 5=trifft völlig zu\end{tabular} & metrisch     \\
\midrule
17-21 & \begin{tabular}[c]{@{}l@{}}Komunikations-\\ kompetenz\\KK1,KK2,KK3,KK4\end{tabular}           & \begin{tabular}[c]{@{}l@{}}Sich mit anderen kreativ auseinander setzen, \\ kommunikativ und selbstorganisiert handeln\\ \parencite[8]{ErpenbeckRosenstiel200305}\end{tabular} & \begin{tabular}[c]{@{}l@{}}5-stufige Likertskala\\ 1= trifft nicht zu,\\2= trifft wenig zu \\ 3=trifft teils/teils zu \\ 4=trifft überwiegend zu \\ 5=trifft vollständig zu\end{tabular}                                                                                    & metrisch     \\
 \midrule
22-28 & \begin{tabular}[c]{@{}l@{}}Nutzungs\\ verhalten\\ Lernplattform NV1,NV2,NV3\\NV4,NV5,NV6,NV7,NV8\end{tabular} & \begin{tabular}[c]{@{}l@{}}Aufschluss über die Intensität u\\ der Nutzung der Lernplattform\end{tabular}                                                                 & \begin{tabular}[c]{@{}l@{}}5-stufige Skala\\ 1=0-1mal genutzt\\ 2=2-4 mal genutzt\\ 3=5-7 mal genutzt\\ 4=8-10 mal genutzt\\ 5=\textgreater10 mal genutzt\end{tabular}                            & metrisch     \\ \bottomrule
\end{tabular}%
}
\end{table}



%\input{tabelleanalysen}
%\includepdf{alleine.pdf}
\end{document}
